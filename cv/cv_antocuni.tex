\documentclass[11pt,a4paper]{moderncv}

% moderncv themes
\moderncvtheme[blue]{casual}  % optional argument are 'blue' (default), 'orange', 'red', 'green', 'grey' and 'roman' (for roman fonts, instead of sans serif fonts)
%\moderncvtheme[green]{classic}                % idem

% character encoding
\usepackage[utf8]{inputenc}

% adjust the page margins
\usepackage[scale=0.8]{geometry}
\recomputelengths % required when changes are made to page layout lengths

% personal data
\firstname{Antonio}
\familyname{Cuni}
\title{Curriculum Vitae} 
\email{info@antocuni.eu}
\extrainfo{http://antocuni.eu}
\photo[64pt]{antocuni} % '64pt' is the height the picture must be resized to
% \quote{
% Considerate la vostra semenza: \\
% fatti non foste a viver come bruti, \\
% ma per seguir virtute e canoscenza. \\
% \emph{(Dante Alighieri)}
% }

\quote{
  Last updated: May 28, 2019
}

%\nopagenumbers{} % uncomment to suppress automatic page numbering for CVs longer than one page


%----------------------------------------------------------------------------------
%            content
%----------------------------------------------------------------------------------
\begin{document}
\maketitle

\section{Personal Information}
\cvitem{Name}{Antonio Cuni}
\cvitem{Date of Birth}{May 13, 1982}
\cvitem{Place of Birth}{Genova, Italy}
\cvitem{Nationality}{Italian}
\cvitem{Email}{info@antocuni.eu}
\cvitem{Website:}{http://antocuni.eu}
\section{Summary}
\cvitem{Introduction}{I have been designing and implementing software systems
 since 1994, first as a hobby, then as a study subject and as a job.  I care
  about building solid and extensible designs, and about writing good and
  readable code. Being a developer, I like to write tools targeted to myself
  and to the others developers, to make our life easier.}

\cvitem{Technologies}{I am a primary a Python expert, but I have a strong
  experience also with other technologies, including (but not limited to)
  Linux, C, C++, SQL and database design, Web development in general, Java,
  .NET and C\#, computer networking and security.  Moreover, since programming
  and computers in general are still a pleasant hobby I like to learn about
  new technologies, and I can usually do it very quickly.  In the latest years
  my expertise on PyPy brought me to work on various tasks to improve the
  performance of existing systems, making me an expert in the field of
  performance, optimization and low-latency systems.}

\cvitem{PyPy and FLOSS}{I have been a core contributor to various aspects of
  PyPy since 2006, including the development of the CLI/.NET and JVM backends,
  the CLI JIT backend, the PyPy JIT compiler generator, the PyPy Python
  interpreter and the testing infrastructure.  Currently, I am interested in
  optimizing the \emph{cpyext} module, with the final goal of making the whole
  Python scientific ecosystem very fast on PyPy.  Other than PyPy, I started
  and have contributed to several FLOSS projects, as detailed below.}

\cvitem{Methodologies}{I am addicted to TDD, and I don't even consider jobs or
  positions for which tests are not important.  I care \emph{a lot} about
  designing beautiful and Pythonic APIs. Thanks to my experience with PyPy, I
  have a strong and positive experience w.r.t. distributed, agile,
  sprint-driven and test-driven development, including good communication
  skills through IRC and emails, the ability to work from home and to be
  self-motivated to accomplish my goals.}


\section{Spoken languages}
\cvlanguage{Italian}{Native speaker}{}
\cvlanguage{English}{Fluent}{Cambridge FIRST Certificate in English}

\section{Education}
\cventry{2007 -- 2010}{Ph.D. in Computer Science}{DISI}{Università degli Studi
  di Genova}{Italy}{}
\cvitem{}{\emph{Dottorato di Ricerca in Informatica}}
\cvitem{}{\textsc{Disseration}}
\cvitem{Title}{\emph{High performance implementation of Python for CLI/.NET with JIT
compiler generator for dynamic languages.}}
\cvitem{Advisor}{Professor Davide Ancona}
\cvitem{}{}

\cventry{2004 -- 2006}{Master in Computer Science}{DISI}{Università degli Studi
  di Genova}{Italy}{}
\cvitem{}{\emph{Laurea Specialistica in Informatica}}
\cvitem{Grade}{$110$ out of $110$ \emph{cum laude}}
\cvitem{}{\textsc{Master Thesis}}
\cvitem{Title}{\emph{Implementing Python on .NET}}
\cvitem{Advisor}{Professor Massimo Ancona}
\cvitem{}{}

\cventry{2001 -- 2004}{Bachelor in Computer Science}{DISI}{Università degli Studi
  di Genova}{Italy}{}
\cvitem{}{\emph{Laurea in Informatica}}
\cvitem{Grade}{$110$ out of $110$ \emph{cum laude}}
\cvitem{}{}

\cventry{1996 -- 2001}{High School Diploma}{Liceo Scientifico L. Lanfranconi}{Voltri}{Italy}{}
\cvitem{Grade}{$100$ out of $100$}
\cvitem{}{}

\section{Employment History}
\cventry{2012 -- now}{Consultant}{Gambit Research}{}{} {My primary task in
  Gambit is to take their existing system and make it fast(er).  I cured the
  switch from CPython to PyPy: after optimizing the code specifically for
  PyPy, I got speedups in the order of 10-20x.  To speed up the time spent in
  inter-process communication, I developed \emph{capnpy}, which is a
  (de)serialization library for Cap'n Proto whose speed ranges from ``very
  fast'' on CPython to ``embarassingly fast'' on PyPy.  More recently, I
  worked extensively on the PyPy's Garbage Collector and more in general on
  memory management techniques to optimize the latency of time-critical
  systems.}

\cventry{2001 -- now}{Freelance software developer and architect}{My own
  consultancy}{}{} {During the years, I have designed and implemented numerous
  software systems of various sizes and complexity, involving several
  different technologies, including (but not limited to): Linux, Python, C,
  PostgreSQL, .NET, C\#, SQL Server.  I also gave consultancies and taught
  courses about Software Design, Advanced Python usage and Test Driven
  Development}
\cvitem{}{}

\cventry{Feb. 2011 -- Sep. 2011}{Researcher/developer}{Open End AB}{Göteborg}{}{This
position involved both scientific research and core development in the PyPy
project, including but not limited to the Just In Time (JIT) Compiler
generator and the testing infrastructure.}
\cvitem{}{}

\cventry{2010 -- 2011}{Researcher}{Heinrich-Heine-Universität}{Düsseldorf}{}{This
position involved scientific research in the context of the PyPy project, in
particular for the development of the Just In Time (JIT) Compiler generator.}
\cvitem{}{}

\cventry{2007 -- 2010}{Ph.D Student}{DISI}{Università degli Studi di
  Genova}{}{My Ph.D research was about the effective implementation of dynamic
  languages for statically typed, object oriented virtual machine.  In
  particular, I implemented the PyPy CLI JIT backend for the .NET virtual
  machine.}
\cvitem{}{}

\cventry{2006 -- 2007}{Researcher}{Heinrich-Heine-Universität}{Düsseldorf}{}{This
  position involved scientific research in the context of the PyPy project, in
  particular for bringing it towards maturity.}

\clearpage
\section{FLOSS activity}
\cvitem{PyPy}{Most of my FLOSS contributions are in the context of the PyPy
  project which I have been a core contributor to since 2006.  PyPy is both an
  implementation of Python in Python and a framework for developing dynamic
  languages, featuring among the others a JIT compiler
  generator. \url{http://pypy.org/}}

\cvitem{PyPy ecosystem}{I also contributes to many projects which are strongly
  related to PyPy and/or have been started by the PyPy team. The two most
  notable ones are \texttt{CFFI} and \texttt{vmprof}, which was started by me.}

\subsection{Projects started/owned by me}
\cvitem{pdb++}{drop-in replacement and an extension of the standard
  \texttt{pdb} command line debugger found in the Python Standard Library.  It
  offers several additional features, including syntax highlighting, full
  screen debugging, and a better user interface.  Despite being a relatively
  simple project, \texttt{pdb++} had an extremely good impact on the
  development experience of me and of the other PyPy developers that started
  using it. }
\cvitem{}{\url{https://github.com/antocuni/pdb/}}\vspace{0.2cm}

\cvitem{fancycompleter}{Colorful and sane TAB completion of Python expressions
  for both the interactive interpreter and the \texttt{pdb++} command
  prompt.}
\cvitem{}{\url{http://bitbucket.org/antocuni/fancycompleter/}}\vspace{0.2cm}

\cvitem{capnpy}{Cap'n Proto serialization for Python. Heavily optimized for
  both CPython and PyPy.}
\cvitem{}{\url{https://github.com/antocuni/capnpy}}\vspace{0.2cm}

\cvitem{wmctrl}{Python module to programmatically control windows inside X}
\cvitem{}{\url{http://bitbucket.org/antocuni/wmctrl/}}\vspace{0.2cm}

\cvitem{vcsdeploy}{Easily deploy updates of your application to your customers
  through your favorite Version Control
  System.}
\cvitem{}{\url{http://bitbucket.org/antocuni/vcsdeploy}}\vspace{0.2cm}

\cvitem{gcalendarlet}{A desklet to show Google Calendar events (for
  \texttt{adesklets}).}
\cvitem{}{\url{http://codespeak.net/svn/user/antocuni/gcalendarlet/}}\vspace{0.2cm}

\cvitem{ln.vfat}{Tool to create simulated hard links in a FAT32 filesystem.}
\cvitem{}{\url{http://codespeak.net/svn/user/antocuni/ln.vfat/}}\vspace{0.2cm}

\cvitem{motion}{Motion detection tool.}
\cvitem{}{\url{http://codespeak.net/svn/user/antocuni/motion/}}

\subsection{Projects I contributed to}
\cvitem{}{The following is a (partial and incomplete) list of FLOSS projects I
  have contributed to (e.g., by sending patches and implementing new
  features):}
\cvitem{py.test}{\url{http://pytest.org/}}
\cvitem{pyrepl}{\url{http://codespeak.net/pyrepl/}}
\cvitem{virtualenv}{\url{http://pypi.python.org/pypi/virtualenv/}}
\cvitem{camelot}{\url{http://www.python-camelot.com/}}


\clearpage
% Publications from a BibTeX file
% \newcommand{\commentout}[1]{}
% \commentout{
% \cite{antocuni_2009} % else the publication section does not show up
% }
\nocite{*}
\bibliographystyle{plain}
\bibliography{publications}       % 'publications' is the name of a BibTeX file

\section{Presentations and talks at international conferences}

\cvitem{EuroSciPy '18, Trento}{Antonio Cuni: \emph{How PyPy can help for high-performance computing}}
\cvitem{PyCon PL '18, Poland}{\textbf{Keynote:} \emph{How PyPy can help for high-performance computing}}
\cvitem{PyCon Italia '18, Firenze}{Antonio Cuni: \emph{The practice of TDD:
    tips\&tricks}}
\cvitem{PyCon ZA '17, Cape Town}{Antonio Cuni: \emph{The practice of TDD:
    tips\&tricks}}
\cvitem{EuroPython '17, Rimini}{Antonio Cuni: \emph{The joy of PyPy JIT:
    Abstractions for Free}}
\cvitem{PyCon Italia '17, Firenze}{Antonio Cuni: \emph{PyPy Status Update}}
\cvitem{EuroPython '15, Bilbao}{Antonio Cuni: \emph{Python and PyPy
    performance (not) for dummies}}
\cvitem{PyCon Italia '15, Firenze}{Antonio Cuni: \emph{PyPy JIT (not) for
    dummies}}
\cvitem{EuroPython '13, Firenze}{Antonio Cuni: \emph{Bug Hunting for Dummies}}
\cvitem{PyCon UK '12, Coventry}{Antonio Cuni: \emph{PyPy JIT under the hood}}
\cvitem{EuroPython '12, Firenze}{Antonio Cuni: \emph{PyPy JIT under the hood}}
\cvitem{EuroPython '12, Firenze}{Antonio Cuni: \emph{Python White Magic}}
\cvitem{EuroPython '12, Firenze}{Antonio Cuni, Armin Rigo: \emph{PyPy: current
    status and the GIL-less future}}
\cvitem{EuroPython '11, Firenze}{Antonio Cuni, Armin Rigo: \emph{PyPy in
    production.}}
\cvitem{EuroPython '11, Firenze}{Antonio Cuni, Armin Rigo: \emph{PyPy hands-on.}}
\cvitem{EuroPython '10, Birmingham}{Amaury Forgeot d'Arc, Antonio Cuni, Armin Rigo: \emph{PyPy 1.3: Status and News.}}
\cvitem{Pycon Italia '10, Firenze}{Antonio Cuni, Armin Rigo: \emph{PyPy 1.2:
  snakes never crawled so fast.} Keynote talk.}
\cvitem{EuroPython '09, Birmingham}{Antonio Cuni, Samuele Pedroni: \emph{PyPy Status Talk.}}
\cvitem{EuroPython '09, Birmingham}{Antonio Cuni, Samuele Pedroni: \emph{PyPy: becoming fast.}}
\cvitem{ICOOOLPS '09, Genova}{Antonio Cuni: \emph{Faster than C\#: efficient implementation of dynamic languages on .NET.}}
\cvitem{Pycon Italia '09, Firenze}{Antonio Cuni: \emph{PyPy's Python Interpreter status.}}
\cvitem{Pycon UK '08, Birmingham}{Antonio Cuni, Maciej Fijalkowski: \emph{PyPy and The Art of Generating Virtual Machines.}}
\cvitem{Pycon Italia '08, Firenze}{Antonio Cuni: \emph{PyPy and The Art of Generating Virtual Machines.}}
\cvitem{DLS '07, Montreal}{Antonio Cuni: \emph{RPython: a Step Towards
    Reconciling Dynamically and Statically Typed Object Oriented Languages.}}
\cvitem{EuroPython '07, Vilnius}{Antonio Cuni, Maciej Fijalkowski: \emph{RPython: Need for speed (C and C\# considered harmful).}}
\cvitem{Pycon Italia '07}{Antonio Cuni: \emph{PyPy 1.0 and beyond.}}


\end{document}

